\section{LEIETAKERS VEDLIKEHOLDSPLIKT}


    \begin{enumerate}

        \item Leietaker plikter å behandle så vel leieobjektet som eiendommen for øvrig med tilbørlig aktsomhet

        \item Det påhviler leietaker å besørge og bekoste vedlikehold av leieobjektet, herunder også ut- og innvendig
        vedlikehold av inngangsdør og port samt innvendig vedlikehold, slik at
        alt er i forskrifts- og håndverksmessig god stand. Vedlikeholdsplikten for leietaker omfatter også
        fornyelse av gulvbelegg og annen oppussing og istandsetting innvendig, herunder
        overflatebehandling av gulv, vegger og tak. Videre omfatter vedlikeholdsplikten de i lokalet synlige rør,
        ledninger og installasjoner tilknyttet forsyning av vann, luft, varme og elektrisitet, og
        ventilasjon. Det påhviler likeledes leietaker å besørge og bekoste vedlikehold og utskiftning
        av inventar og utstyr som medfølger leieobjektet. Utleier har ikke ansvar for vedlikehold eller utskiftning
        av innretninger anbragt i leieobjektet av leietaker. Alt arbeid leietaker plikter å utføre, skal han foreta uten
        ugrunnet opphold, med normale intervaller i leieperioden og på en håndverksmessig god måte

        \item Leietakers vedlikeholdsplikt omfatter også skader etter innbrudd og/eller hærverk i leieobjektet,
        herunder skader på inngangsdør og port

        \item Leietaker plikter å sørge for reparasjon og vedlikehold av de skilt etc. som utleier har gitt tillatelse til å
        sette opp

        \item Oppfyller ikke leietaker sin vedlikeholdsplikt er utleier berettiget til, etter skriftlig varsel med 14
        dagers oppfyllelsesfrist, å utføre vedlikeholdsarbeidene for leietakers regning



    \end{enumerate}
