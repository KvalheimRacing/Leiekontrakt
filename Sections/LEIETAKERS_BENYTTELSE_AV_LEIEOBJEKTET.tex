\section{LEIETAKERS BENYTTELSE AV LEIEOBJEKTET}


    \begin{enumerate}


        \item Leietaker plikter å sette seg inn i og følge de offentlige forskrifter, vedtekter, instrukser, ordensregler
        o.l. som er eller måtte bli innført og som kommer til anvendelse på leieforholdet. Leietaker er ansvarlig
        overfor alle offentlige myndigheter for at hans benyttelse av leieobjektet tilfredsstiller de til enhver tid
        gjeldende offentligrettslige krav. Alle offentligrettslige krav, herunder krav fra arbeidstilsyn, helseråd,
        sivilforsvar, industrivern, brannvern, statens vegvesen eller annen offentlig myndighet, foranlediget av den virksomhet som
        drives i leieobjektet, er det leietakers ansvar å oppfylle per overtakelse og for øvrig i leieperioden

        \item Leieobjektet må ikke benyttes på en måte som forringer eiendommens omdømme eller utseende eller
        ved støv, støy, lukt, rystelse eller på annen måte sjenerer andre leietakere eller naboer. Kostnadene ved utbedring og eventuell
        erstatning i forbindelse med disse forhold, er leietakers ansvar

        \item Det henvises til normale regler om ro etter kl 20:00. Det kan drives virksomhet hele døgnet, men etter kl 20:00 bes det om å bruke skjønn med tanke på bråk og støy, spesielt av dype frekvenser. Det er ikke lov å drive med motortuning og støyfull motortesting mellom kl 20:00 og kl 08:00

        \item Det er ikke anledning til å koble på flere elektriske ovner enn det som allerede finnes i lokalet. Ved behov for mer varme eksempelvis som følge av åpen port på vinterstid, må det benyttes jetvarmer eller annen varmer på eget drivstoff. Denne skal ikke stå på uten oppsyn

        \item På vinteren må leietaker selv stå for måking. Det blir satt ut snøskuffer utenfor bygningen som kan benyttes. Det skal ikke måkes snøhauger foran containerne eller utleiers område

        \item Urinering på eiendommens område er ikke tillatt

        \item Tildekning av ovner er ikke tillatt

        \item Avfall må legges i lokalets søppelkasse. Hverken søppelkassen, løst søppel, eller deler fra leietaker skal befinne seg utenfor lokalet.
        Det skal benyttes gjennomsiktige søppelposer som dagen før avfallshenting kan settes ved nabolagets søppelkasser. Avfallet gjelder som restavfall. Appen "Min Renovasjon" må benyttes for informasjon og varsling om lokalets tømmekalender. Det er ikke anledning til å sette mer enn én full søppelpose for henting per tømmedag. Søppelposen skal knytes godt igjen.
        Avfall utover én full søppelsekk eller av ekstraordinært omfang, må leietaker selv besørge fjernet for egen regning.
        Leietaker må selv sørge for tømming av oljefat når dette er fullt, og ved utflyning av lokalet

        \item Røyking er forbudt inne i lokalet. Ved røyking utenfor lokalet skal port samt hoveddør være lukket og askebeger på veggen skal benyttes for sneiper. Leietaker eller leietakers bekjente skal aldri kaste løse sneiper eller snusposer rundt omkring på eiendommen


    \end{enumerate}
