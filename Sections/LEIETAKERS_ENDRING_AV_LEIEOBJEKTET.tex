\section{LEIETAKERS ENDRING AV LEIEOBJEKTET}


    \begin{enumerate}


        \item Leietaker kan ikke foreta bygningsmessig forandring i eller av leieobjektet uten
        utleiers skriftlige forhåndssamtykke, som også kreves om leietaker ønsker å bruke mer strøm, vann,  m.v. enn
        hva leieobjektet ved kontraktstidspunktet var utstyrt med. Samtykke kan ikke nektes uten saklig grunn.
        Spesielt kan man på ingen måte endre sammensetningen av pallereolene

        \item Radio- og høytaler-anlegg m.v., store skap, automater o.l. må ikke settes opp uten utleiers skriftlige
        forhåndssamtykke. Samtykke kan nektes på fritt grunnlag

        \item Endringsarbeider beskrevet i dette punkt 13 tilfaller utleier etter endt leieperiode, med mindre utleier
        forlanger leieobjektet satt tilbake i sin opprinnelige stand

        \item Leietaker er ansvarlig for å innhente de nødvendige offentlige tillatelser for eventuelle arbeider som
        utføres i henhold til dette punkt 13


    \end{enumerate}
