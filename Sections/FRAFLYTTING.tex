\section{FRAFLYTTING}


    \begin{enumerate}

        \item Ved fraflytting skal utleier umiddelbart gis adgang til leieobjektet

        \item Leietaker skal ved fraflytting tilbakelevere leieobjektet ryddiggjort, rengjort og
        for øvrig i kontrakts- og håndverksmessig godt vedlikeholdt stand.
        Dersom leietakers vedlikeholdsplikt er oppfylt med alminnelige intervaller i leieperioden,
        aksepterer utleier normal slit og elde frem til fraflytting. Hvor annet ikke er avtalt i forbindelse med
        leietakers endringsarbeider (se punkt 15) skal fast inventar, delevegger, ledninger o.l. ikke fjernes ved
        fraflytting, men tilfalle utleier uten godtgjørelse. Utleier kan kreve at leietaker ved fraflytting fjerner helt
        eller delvis leietakers endringsarbeider, herunder innredning og innretninger, ledninger o.a. han har
        montert i leieobjektet, og at skader og merker som følge av dette utbedres. Oppfylles ikke disse plikter,
        kan utleier utføre arbeidet for leietakers regning

        \item Mangler som leietaker ikke har utbedret, kan utleier la utbedre for leietakers regning

        \item I god tid før leieforholdets opphør skal det avholdes en felles befaring mellom leietaker og utleier for å
        fastlegge eventuelt nødvendige arbeider for å bringe leieobjektet i den stand det skal være ved
        tilbakelevering

        \item I de siste 5 måneder før fraflytting har utleier rett til å legge ut annonse på finn, med informasjon om
        at leieobjektet blir ledig. I samme periode plikter leietaker, etter forhåndsvarsel, å gi leiesøkende adgang
        til leieobjektet 2 dager pr. uke i kontor/forretningstid

        \item Senest siste dag av leieforholdet skal leietaker på egen bekostning fjerne sine eiendeler. Eiendeler som
        ikke fjernes skal anses etterlatt, og tilfaller utleier etter 14 dager. Søppel og eiendeler som utleier ikke
        ønsker å overta kan utleier kaste eller fjerne for leietakers regning


    \end{enumerate}
