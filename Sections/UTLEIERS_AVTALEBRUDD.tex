\section{UTLEIERS AVTALEBRUDD}


    \begin{enumerate}


        \item Leietaker kan kreve avslag i leien i henhold til husleieloven § 2-11 som følge av forsinkelse eller
        mangel. Hva gjelder mangel, forutsettes at mangelen er vesentlig og at mangelen ikke rettes av utleier i henhold
        til bestemmelsene i husleieloven § 2-10. Denne bestemmelse gjelder både
        forsinkelse/mangler pr. overtakelse og mangler i leietiden. Leietaker må gi skriftlig melding om skader og mangler
        mv. innen rimelig tid etter at leietaker burde ha oppdaget dem. Rimelig tid er definert i punkt 5.2

        \item Leietaker har ikke rett til å holde tilbake leie til sikkerhet for de krav leietaker har eller måtte få mot utleier som følge av mangel eller forsinkelse

        \item Leietaker kan kreve erstatning for direkte tap som følge av forsinkelse eller mangel i henhold til
        husleieloven § 2-13. Hva gjelder mangel, forutsettes at mangelen er vesentlig og at mangelen ikke rettes av
        utleier i henhold til bestemmelsene i husleieloven § 2-10. Indirekte tap dekkes ikke.
        Erstatningens størrelse begrenses oppad til ett kvartals leie, med mindre utleier har handlet forsettlig
        eller grovt uaktsomt. Denne bestemmelse gjelder både forsinkelse/mangler pr. overtakelse og mangler i
        leietiden

        \item Dersom leietaker ønsker å påberope vedvarende eller gjentatt mislighold fra utleiers side som grunnlag
        for heving, se for øvrig kravene i husleieloven § 2-12, krever dette skriftlig forhåndsvarsling om at avtalen
        kan bli hevet om misligholdet ikke opphører


    \end{enumerate}
