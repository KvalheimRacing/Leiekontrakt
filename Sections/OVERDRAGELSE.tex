\section{OVERDRAGELSE}


    \begin{enumerate}

        \item Leietaker kan ved behov for utflyning før leieperiodens slutt, finne kandidater til overdragelse av leieavtalen

        \item Overdragelse av leieavtalen, helt eller delvis er ikke tillatt uten utleiers skriftlige forhåndssamtykke.
        Samtykke kan nektes på fritt grunnlag

        \item Overdragelse av minst 50\% av aksjene, selskapsandelene eller eierinteressene hos leietaker anses som
        overdragelse av leieavtalen. Det samme gjelder leietakers skifte av selskapsform
        %Som overdragelse regnes også avhendelse av det mindre antall aksjer eller andeler som i seg selv utgjør bestemmende
        %innflytelse (alminnelig flertall) i selskapet. Utleier skal på forespørsel gis opplysninger bekreftet av
        %leietakers revisor, dersom han ønsker å kontrollere om slik overdragelse har funnet sted.

        \item Selskapsmessige endringer, eksempelvis fisjoner, som kan forringe leietakers økonomiske stilling
        overfor utleier, krever utleiers skriftlige samtykke

        \item Manglende svar på søknad om samtykke etter bestemmelsene i dette punkt 21 anses ikke som
        samtykke

        \item Dersom utleier overfører sine rettigheter og forpliktelser etter avtalen skal leietaker medvirke til at nytt depositum stilles overfor ny eier

    \end{enumerate}
