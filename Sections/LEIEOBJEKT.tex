\section{LEIEOBJEKT}


	\begin{enumerate}


		\item Arealer til leietakers bruk er: ca. 80 kvm. BTA

		\item Eventuelle feil i arealangivelsene gir ikke rett til å kreve leien justert, og medfører heller ikke noen endring av
		denne leieavtales øvrige bestemmelser

		\item Det er stor plass for flere midlertidige parkeringer utenfor lokalet. Her gjelder for leietaker "parkering på eget ansvar", da utleier leier ut containere som benyttes til lagerplass, og dersom leietakere av containerne trenger tilgang til disse må dette gis umiddelbart. Leietaker oppfordres til å leie en eller flere containere for å skaffe seg faste parkeringsplasser på området

		\item Det er felles snuplassområder i nabolaget, og leietaker står fritt til å snakke med naboer i forhold til å bruke denne til midlertidige parkeringsplasser, da dette har blitt gjort av tidligere leietakere

		\item Følgende er mulige utvidelser som utleier vil dekke 50\% av kostnaden på, mot at utvidelsene tilfaller leieobjektet og utleiers eie:

			\begin{enumerate}

				\item Installasjon i lokalet av skruekompressor m. lufttørke, rørsystem, slangetromler hengende fra taket og luftuttak langs veggen. Dette systemet vil forsyne begge verkstedlokalene som er vegg i vegg

				\item Bygging av sliperom under hems. Det er tiltenkt at man tetter igjen åpningen mot reolene og henger opp svart sveiseforheng med gardintype oppheng for å ha muligheten til å dekke åpningen mot løftebukken. Installasjon av "disk-sander", båndsliper, skrustikke og slipebord

				\item Generell utbedring av ventilasjonsanlegg og spesielt tillegg med egen eksosventilasjon

				\item Materialkappesag for alt av metaller

				\item Sandblåsekabinett med installasjon

				\item Hydraulisk presse

				\item Søylebormaskin

				\item Plasmakutter

			\end{enumerate}

			Disse utvidelsene vil projekteres og styres av utleier, og leietaker plikter og sette seg inn i og benytte seg av brukerinstrukser og manualer for medfølgende utstyr. Hvorvidt utvidelsene igangsettes eller ikke avgjøres av utleier, dersom leietaker ønsker en utvidelse

		\item Leieobjektet er vegg i vegg med utleiers eget verksted, og det benyttes en felles hoveddør

		\item På sommeren er det anledning til å benytte seg av bygningens utekran for varmt og kaldt vann, for vasking av biler eller lignende. Det fins stikkontakt utenfor til bruk av f.eks høytrykkspyler

		\item Inkludert i leieobjektet er følgende:

			\begin{itemize}

				\item 3 Seter skinnsofa

				\item Whiteboardtavle

				\item Veggmonterte ovner på tilsammen 2,4kW

				\item Wi-Fi. Hastighet og stabilitet kan variere som følge av nettverksproblemer fra teleselskapet

				\item 42" TV og Chromecast

				\item Brannslukningsaparater

				\item Alarmanlegg med brannvarsler og vekterutrykning

				\item Elektronisk kodelås med app

				\item Punktavsug

				\item Metallhylle

				\item Jekketralle

				\item Reoler

			\end{itemize}

		\item Vask og kjøleskap i utleiers verksted kan benyttes. Det er ikke anledning for å låne noe, ta noe, eller benytte seg av noe som helst annet i utleiers eget verksted. Det spesifiseres at dette godet forutsetter en gjensigig tillit og god kommunikasjon. Dersom utleier vurderer det dithen at dette ikke forekommer kan leietaker miste tilgangen til bruk av vask og kjøleskap. Dette vil f.eks forekomme dersom verktøy eller utstyr blir stålet, vask eller kjøleskap tilgriset uten vilje til å rengjøre, eller ved oppdagelse av personer på andre steder som de ikke skal være i utleiers eget verksted. Vask og kjøleskap skal etter hver gangs bruk forlates i like god eller renere tilstand enn før man benyttet det

		\item Toalett medfølger ikke, men det er i utleiers kontorbygg et toalett. Mellom kl 8 og kl 18 er det anledning til å banke på døren og spørre om å låne toalettet. Dersom toalettet forlates i like god eller renere tilstand er det svært liten sansynlighet for å få et "nei" ved neste trengende anledning

		\item Varmen i lokalet er den som kan genereres av de tre ovnene på tilsammen 2,4kW

	\end{enumerate}
