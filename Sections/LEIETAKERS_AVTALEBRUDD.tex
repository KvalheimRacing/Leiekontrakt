\section{LEIETAKERS AVTALEBRUDD}


    \begin{enumerate}


        \item Leietaker blir erstatningsansvarlig for all skade eller mangler som skyldes ham selv eller folk i hans
        tjeneste, faste eller tilfeldige, samt fremleietakere, kunder, leverandører og/eller andre personer som han
        har gitt adgang til eiendommen. Erstatningsplikten omfatter også utgift som måtte følge av utrydding av
        utøy

        \item Leietaker vedtar at tvangsfravikelse kan kreves hvis leien eller avtalte tilleggsytelser ikke blir betalt, jf.
        § 13-2 3. ledd (a) i tvangsfullbyrdelsesloven. Leietaker vedtar at tvangsfravikelse kan kreves når leietiden
        er løpt ut, jf. § 13-2 3. ledd (b) i tvangsfullbyrdelsesloven. Ved vesentlig mislighold som ikke gjelder manglende betaling og fraflytting kan utleier benytte seg av lov om tvangsfullbyrdelse § 13-2, 3. ledd pkt. c, d og e

        \item Gjør leietaker seg skyldig i vesentlig mislighold av leieavtalen kan utleier heve denne, og leietaker
        plikter da å fraflytte leieobjektet

        \item En leietaker som blir kastet ut eller flytter etter krav fra utleier pga. mislighold eller fraviker
        leieobjektet som følge av konkurs, plikter å betale leie (og eventuelle øvrige betalingsforpliktelse under
        leieavtalen) for den tid som måtte være igjen av leietiden. Betalingsplikten suspenderes for den periode
        utleier får leid ut leieobjektet på ny, til samme eller høyere pris. Leietaker må også betale de
        omkostninger som utkastelse, søksmål og rydding/rengjøring av leieobjektet fører med seg, samt utgifter
        til ny utleie


    \end{enumerate}
